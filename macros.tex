\newcommand*\circled[1]{\tikz[baseline=(char.base)]{
  \node[shape=circle,draw,fill=black,text=white,font=\bf,inner sep=0.5pt] (char)
  {\scriptsize#1};
}}

\newcommand*\circledwhite[1]{\tikz[baseline=(char.base)]{
  \node[shape=circle,draw,fill=white,text=black,font=\bf,inner sep=0.5pt] (char)
  {\scriptsize#1};
}}

\newcommand{\putsec}[2]{\vspace{-0.0in}\section{#2}\label{sec:#1}\vspace{-0.0in}}
\newcommand{\putssec}[2]{\vspace{-0.0in}\subsection{#2}\label{ssec:#1}\vspace{-0.0in}}
\newcommand{\putsssec}[2]{\vspace{-0.0in}\subsubsection{#2}\label{sssec:#1}\vspace{-0.0in}}
%\newcommand{\putsssecX}[1]{\vspace{0.0in}\subsubsection*{#1}\vspace{0.0in}}
\newcommand{\putsssecX}[1]{\vspace{0.0in}\noindent\textbf{#1:}}

\newcommand{\figref}[1]{Figure~\ref{fig:#1}}
\newcommand{\eqnref}[1]{Equation~\ref{eq:#1}}
\newcommand{\tabref}[1]{Table~\ref{tab:#1}}
\newcommand{\secref}[1]{Section~\ref{sec:#1}}
\newcommand{\ssecref}[1]{Section~\ref{ssec:#1}}
\newcommand{\sssecref}[1]{Section~\ref{sssec:#1}}

\newcommand{\COMMENT}[1]{#1}
\newcommand{\prash}[1]{\COMMENT{\textcolor{blue}{\sf Prash: #1}}}
\newcommand{\ashu}[1]{\COMMENT{\textcolor{green}{\sf Ashutosh: #1}}}
\newcommand{\nachi}[1]{\COMMENT{\textcolor{green}{\sf Nachi: #1}}}
\newcommand{\jash}[1]{\COMMENT{\textcolor{green}{\sf Jashwant: #1}}}
\newcommand{\das}[1]{\COMMENT{\textcolor{red}{\sf Dr.Das: #1}}}
\newcommand{\mahmut}[1]{\COMMENT{\textcolor{red}{\sf Dr.Kandemir: #1}}}
\newcommand\pras[1]{\textcolor{Green}{{\bf Prasanna:} #1}}
\newcommand\bsharma[1]{\textcolor{red}{{\bf Bikash:} #1}}
\newcommand{\todo}[1]{\COMMENT{{\color{red}\sf\bfseries #1}}}
\makeatletter
\renewcommand{\p@subsection}{}
\renewcommand{\p@subsubsection}{}
\makeatother

\renewcommand{\footnoterule}{
  \kern -3pt
  \hrule width 0.49\textwidth height 1pt
  \kern 2pt
}

\newcommand{\cmark}{\ding{51}}
\newcommand{\xmark}{\ding{55}}
\newcommand{\ignore}[1]{}

% Uncomment this to remove the inline comments
%\renewcommand{\COMMENT}[1]{}
